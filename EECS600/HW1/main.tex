\documentclass[letterpaper]{article}
\usepackage{amssymb}
\usepackage{fullpage}
\usepackage{changepage}
\usepackage{amsmath}
\usepackage{epsfig,float,alltt}
\usepackage{psfrag,xr}
\usepackage[T1]{fontenc}
\usepackage{url}
\usepackage{pdfpages}
\usepackage{epstopdf}
\usepackage[framed,numbered,autolinebreaks,useliterate]{mcode}

%\includepdfset{pagecommand=\thispagestyle{fancy}}
\author{Yi Yang}
\title{Solution of EECS 600 HW \#1}

\begin{document}
\date{}
\maketitle

\newcommand{\trace}{\mathrm{trace}}
\newcommand{\real}{\mathbb R}  % real numbers  {I\!\!R}
\newcommand{\nat}{\mathbb N}   % Natural numbers {I\!\!N}
\newcommand{\cp}{\mathbb C}    % complex numbers  {I\!\!\!\!C}
\newcommand{\ds}{\displaystyle}
\newcommand{\mf}[2]{\frac{\ds #1}{\ds #2}}
\newcommand{\spanof}[1]{\textrm{span} \{ #1 \}}
\newcommand{\sol}[0]{\textbf{Solution: }}
\newcommand{\pf}[0]{\textbf{Proof: }}
\newcommand{\rme}[0]{\textrm{e}}
\newcommand{\Null}[1]{\textrm{Null}\{#1\}}
\parindent 0pt

\begin{enumerate}
	\item[1.]\pf
	The collection of infinite number of sequences which converges to zero is a vector space. In order to prove this statement, we should see whether it satisfies $5$ addition and $5$ scalar multiplication conditions:\\
	\large{\textbf{Addition}}:\\
	\textbf{Closure: }$\forall\;\{x_n\}_{n=1}^\infty, \{y_n\}_{n=1}^\infty\in \textrm{X}$, and we want to see $\{x_n + y_n\}_{n=1}^\infty\in\textrm{X}$\;.\\
	$\forall\;\epsilon = \epsilon_{1}+\epsilon_{2} > 0, \epsilon_{1}>0, \epsilon_{2}>0$, there exists $N>0$, such that $|x_n + y_n|<\epsilon$ for all $n>N$\;.\\
	From the convergence of sequences $\{x_n\}$ and $\{y_n\}$, we know the following statements are true:\\
	$\forall \epsilon_{1}, \epsilon_{2} > 0$, there exists $N_{1}$ and $N_{2}$ that are both greater than $0$, such that $|x_n|<\epsilon_{1}$ for all $n > N_{1}$ and $|y_n|<\epsilon_{2}$ for all $n>N_{2}$, we choose $N=\max\{N_{1},N_{2}\}$ so that we know the convergence statement satisfies for the sum of two sequences\;.\\
	\textbf{Commutativity, Associativity: } can be deduced from commutativity and associativity of the additions of real numbers\;.\\
	\textbf{Additive identity: } $\exists\;\{0\}_{n=1}^{\infty}$ such that $\forall\;\{x_n\}_{n=1}^{\infty}\in\textrm{X}$, $\{0+x_n\}_{n=1}^{\infty}=\{x_n\}_{n=1}^{\infty}\in\textrm{X}$\;.\\
	\textbf{Additive inverse: } additive inverse of two infinite sequences of real numbers is a infinite sequences filled with $0$ which satisfies our condition\;.\\
	\large{\textbf{Scalar Multiplication}}:\\
	\textbf{Closure: } $\forall\;\epsilon=\mf{\epsilon_{1}}{|a|}>0,\;a\in\real$, there exists $N>0$, such that $|\mf{x_n}{a}|<\mf{\epsilon_{1}}{|a|}$ for all $n>N$\;.\\ 
	\textbf{Associativity, First Distributivity, Second Distributivity and application of multiplicative identity} can be deduced from properties of real numbers and closure conditions.\\
	
	\item[2.]
	
	
	
    \item[3(a)]\sol
The matlab code for both versions is attached below.
\lstinputlisting[firstline = 1, lastline = 67]{problem3.m}
For the general case, we can get $A$ is:
  \begin{equation*}
    A=\begin{pmatrix}
        8.000000000000000&  -4.582575694955841&   0&   0\\
      -4.582575694955841&  12.571428571428569 &  3.886134431067267 & -0.000000000000001\\
      0 &  3.886134431067267 &  7.666409266409263 & -1.188992612745301\\
      0  &-0.000000000000001 & -1.188992612745300 &  3.762162162162162
    \end{pmatrix},
  \end{equation*}
For the real symmetric case:
    \begin{equation*}
    A=\begin{pmatrix}
           8.000000000000000 & -4.582575694955840     &              0       &            0\\
            -4.582575694955840  &12.571428571428562  & 3.886134431067272     &              0\\
                   0  & 3.886134431067272  & 7.666409266409261 & -1.188992612745300\\
                   0    &               0 & -1.188992612745300 &  3.762162162162163
    \end{pmatrix}.
  \end{equation*}

  \item[3(b)]
  The cputime for the two methods are listed below.
  \begin{table}[H]
    \centering
    \begin{tabular}{|c|c|c|}
    \hline
    % after \\: \hline or \cline{col1-col2} \cline{col3-col4} ...
    $m$ & general algorithm & symmetric\\ \hline
    100 &            0      & 0.218401400000005\\
    200 & 0.702004500000001 &  1.310408400000000\\
    300 & 0.826805299999997 &  3.915625100000000\\
    400 & 2.090413400000003 &  5.007632099999995\\
    500 & 4.040425900000003 &  8.751656100000005\\
    600 & 7.971651099999988 & 14.086890300000007\\
    700 & 14.367692099999999&  21.746539400000003\\
    800 & 24.133354699999984&  31.746203500000007 \\   
    900 & 36.410633399999995&  43.664679899999982\\
    1000& 52.853138800000011&  59.997984599999995\\ \hline
  \end{tabular}\caption{CPU time.\label{tab:pb3_1}}
  \end{table}
  From this table, we can see the general case will cost less cputime to run the code. This is because Matlab will spend extra time to run the inner loops.
  \item[4(a)]\sol
Matlab code is listed below.
    \lstinputlisting[firstline = 1, lastline = 100]{problrm4.m}
    The results at each step are\\[.2 cm]
    \begin{tabular}{|c|c|c|c|}
      \hline
      % after \\: \hline or \cline{col1-col2} \cline{col3-col4} ...
      Iteration Number & Power method & inverse iteration & Rayleigh quotient\\\hline
       1 &  16.672131147540984 & 14.399999999999999 & 16.684615384615384\\
       2 &  16.683819628647210 &  7.135135135135138 & 16.684658438426489\\
       3 &  16.684602322430148 &  4.554938956714761 & 16.684658438426489\\
       4 &  16.684654684513276 &  4.331664534989951 & 16.684658438426489\\
       5 &  16.684658187307214 &  4.316434838256361 & 16.684658438426489\\
       6 &  16.684658421627791 &  4.315414702695186 & 16.684658438426489\\
       7 &  16.684658437302737 &  4.315346454398972 & 16.684658438426489\\
       8 &  16.684658438351320 &  4.315341888880738 & 16.684658438426489\\
       9 &  16.684658438421469 &  4.315341583468829 & 16.684658438426489\\
      10 &  16.684658438426155 &  4.315341563038205 & 16.684658438426489\\ \hline
    \end{tabular}\\

  \item[4(b)]\sol The maximum magnitude of the off-diagonal elements at each step is\\
  \begin{table}[!h]
    \centering
    \begin{tabular}{|c|c|}
    \hline
    % after \\: \hline or \cline{col1-col2} \cline{col3-col4} ...
    Iteration Number & maximum magnitude of off-diagonal element\\ \hline
    1 & 4.289906995109921\\
    2 & 2.659558449077073\\
    3 & 1.363329400421120\\
    4 & 0.727713493424353\\
    5 & 0.394169759053324\\
    6 & 0.252862471417952\\
    7 & 0.178747433831389\\
    8 & 0.125365640161673\\
    9 & 0.087544529634242\\
   10 & 0.060995911929791\\ \hline
  \end{tabular}
  \end{table}\\[.2cm]
  and the final matrix is
  \begin{equation*}
    A^{(10)}=
    \begin{pmatrix}
    16.684650870697748 &  0.008106965374877 &  0.000018128314708 &  0.000000256400932\\
    0.008106965374876  & 7.999920721769701  & 0.017888947809789  & 0.000253015404780\\
    0.000018128314708  & 0.017888947809787  & 4.312593875744196  & 0.060995911929791\\
    0.000000256400932  & 0.000253015404780  & 0.060995911929791  & 3.002834531788369
    \end{pmatrix}.
  \end{equation*}\\

\end{enumerate}
\end{document}}
