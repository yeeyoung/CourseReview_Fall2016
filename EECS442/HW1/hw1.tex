\documentclass[letterpaper]{article}
\usepackage{amssymb}
\usepackage{fullpage}
\usepackage{changepage}
\usepackage{amsmath}
\usepackage{epsfig,float,alltt}
\usepackage{psfrag,xr}
\usepackage[T1]{fontenc}
\usepackage{url}
\usepackage{pdfpages}
\usepackage{epstopdf}
\usepackage[framed,numbered,autolinebreaks,useliterate]{mcode}

%\includepdfset{pagecommand=\thispagestyle{fancy}}
\author{Yi Yang}
\title{EECS 442 Homework \#1}

\begin{document}
\date{09/17/2016}
\maketitle

\newcommand{\trace}{\mathrm{trace}}
\newcommand{\real}{\mathbb R}  % real numbers  {I\!\!R}
\newcommand{\nat}{\mathbb N}   % Natural numbers {I\!\!N}
\newcommand{\cp}{\mathbb C}    % complex numbers  {I\!\!\!\!C}
\newcommand{\ds}{\displaystyle}
\newcommand{\mf}[2]{\frac{\ds #1}{\ds #2}}
\newcommand{\spanof}[1]{\textrm{span} \{ #1 \}}
\newcommand{\sol}[0]{\textbf{Solution: }}
\newcommand{\pf}[0]{\textbf{Proof:}}
\newcommand{\rme}[0]{\textrm{e}}
\newcommand{\Null}[1]{\textrm{Null}\{#1\}}
\parindent 0pt
%%%%%%%%%%%%%%%%%%%%%%%%%%%%%%%%%%%%%%%%%%%%%%%%%%%%%%%%%%%%%%%%%%%%%%%%%%%%%%%
% Solution for Question 1 begins here - by Yi Yang
%%%%%%%%%%%%%%%%%%%%%%%%%%%%%%%%%%%%%%%%%%%%%%%%%%%%%%%%%%%%%%%%%%%%%%%%%%%%%%%
\section*{Problem 1}
\subsection*{(a)}
Suppose $X_c$ and $X_w$ are the same coordinates represented in Camera and World coordinates. Hence, we have the relation $X_c = PX_w$, $P$ is the $4\times4$ combined rotation and translation matrix from homogeneous world coordinates to homogeneous camera coordinates. It is easy to get following relations:
$$W_x = -\cos(45^{\circ})\times C_x + 0\times C_y - \cos(45^{\circ})\times C_z$$
$$W_y = C_y$$
$$W_z = \cos(45^{\circ})\times C_x + 0\times C_y - \cos(45^{\circ})\times C_z$$
Hence, the transformation matrix:\\
$$ P = 
\begin{bmatrix}
	    -\cos(45^{\circ}) & 0 & \cos(45^{\circ}) & 0\\
		0 & 1 & 0 & 0\\
		-\cos(45^{\circ}) & 0 & -\cos(45^{\circ}) & d\\
		0 & 0 & 0 & 1\\
\end{bmatrix}
=
\begin{bmatrix}
		-\mf{\sqrt{2}}{2} & 0 & \mf{\sqrt{2}}{2} & 0\\
		0 & 1 & 0 & 0\\
		-\mf{\sqrt{2}}{2} & 0 & -\mf{\sqrt{2}}{2} & d\\
		0 & 0 & 0 & 1\\
\end{bmatrix}
$$
\subsection*{(b)}
Consider the unit square in the world coordinate and transform it into camera coordinate, suppose $a^w = [0 \; p \; q \; 1]^T$, $b^w = [0 \; 	p+1 \; q \; 1]^T$, $c^w = [0 \; p+1 \; q+1 \; 1]^T$ and $d = [0 \; p \; q+1 \; 1]^T$. After transformation, the new coordinates for four points of the square are: $a^c = [\mf{\sqrt{2}}{2}q ,\; p, \; -\mf{\sqrt{2}}{2}q, \; 1]^T$, $b^c = [\mf{\sqrt{2}}{2}q ,\; p+1, \; -\mf{\sqrt{2}}{2}q, \; 1]^T$, $c^c = [\mf{\sqrt{2}}{2}(q+1) ,\; p+1, \; -\mf{\sqrt{2}}{2}(q+1), \; 1]^T$ and $d^c = [\mf{\sqrt{2}}{2}(q+1) ,\; p, \; -\mf{\sqrt{2}}{2}(q+1), \; 1]^T$. From these coordinates we see the transformation will not change the form of the square, thereby the area of the the square will remain the same.
\subsection*{(c)}
Assume two end point of one line is $a^w = [x_1,\; y_1,\; z_1,\; 1]$, $b^w = [x_1+p,\; y_1+q,\; z_1+r,\; 1]$ and the end points of the parallel line are $c^w = [x_2,\; y_2,\; z_2,\; 1]$, $d^w = [x_2+pt,\; y_2+qt,\; z_2+rt,\; 1]$. After the transformation, we get the new coordinates for new end points of these two parallel lines:
$$a^c = [-\mf{\sqrt{2}}{2}x_1 + \mf{\sqrt{2}}{2}z_1,\; y_1,\; -\mf{\sqrt{2}}{2}x_1 - \mf{\sqrt{2}}{2}z_1 + d,\; 1]^T$$
$$b^c = [-\mf{\sqrt{2}}{2}(x_1 + p) + \mf{\sqrt{2}}{2}(z_1 + r),\; y_1 + q,\; -\mf{\sqrt{2}}{2}(x_1 + p) - \mf{\sqrt{2}}{2}(z_1 + r) + d,\; 1]^T$$
$$c^c = [-\mf{\sqrt{2}}{2}x_2 + \mf{\sqrt{2}}{2}z_2,\; y_2,\; -\mf{\sqrt{2}}{2}x_2 - \mf{\sqrt{2}}{2}z_2 + d,\; 1]^T$$
$$b^c = [-\mf{\sqrt{2}}{2}(x_2 + pt) + \mf{\sqrt{2}}{2}(z_2 + rt),\; y_2 + qt,\; -\mf{\sqrt{2}}{2}(x_2 + pt) - \mf{\sqrt{2}}{2}(z_2 + rt) + d,\; 1]^T$$
We can easily get the following relations:
$$\overrightarrow{a^cb^c} = [\mf{\sqrt{2}}{2}(r - p),\; q,\; -\mf{\sqrt{2}}{2}(r + p), 0]^T$$
$$\overrightarrow{c^cd^c} = [\mf{\sqrt{2}}{2}(r - p)t,\; qt,\; -\mf{\sqrt{2}}{2}(r + p)t, 0]^T$$
Hence, we can conclude that parallel lines in the world reference system still parallel in the camera reference system as proven above.
\subsection*{(d)}
$$\overrightarrow{a^wb^w} = [0,\; 1,\; 0,\; 0]^T$$
$$\overrightarrow{a^cb^c} = [0,\; 1,\; 0,\; 0]^T$$
we can easily conclude that vector defined by $a$ and $c$ have the same orientation in both reference system.

\hspace*{2em}
\section*{Problem 2}
The projection point $P^\prime = [x^\prime,\; y^\prime]^T$ can be calculated by:
\begin{align*}
	x^\prime = & f^\prime \mf{x}{z}\\
	y^\prime = & f^\prime \mf{y}{z}
\end{align*}
Hence, the projection of points on line are given by
$x^\prime = f^\prime \mf{1}{t}$ and $y^\prime = f^\prime \mf{1}{t}$, we see the end points of the original line requires t equal to $-\infty$ and $-1$, so we can get the end points of the projection are $[-f^\prime,\; -f^\prime]^T$ and $[0,\; 0]^T$.

\hspace*{2em}
\section*{Problem 3}
\subsection*{(a)}

\end{document}